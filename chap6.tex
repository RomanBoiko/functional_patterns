\chapter{Conclusions}
\label{chap:conclusions}
That's all folks!

\begin{center}
\texttt{<http://www.lipsum.com/>}
\end{center}

to the file \texttt{6-Bibliography.bib}

~\cite{res1}
~\cite{res2}
~\cite{res3}
~\cite{res4}
~\cite{res5}
~\cite{res6}
~\cite{res7}
~\cite{res8}
~\cite{res9}
~\cite{res10}
~\cite{res11}
~\cite{res12}
~\cite{res13}
~\cite{res14}
~\cite{res15}
~\cite{res16}
~\cite{res17}
~\cite{res18}
~\cite{res19}
~\cite{res20}
~\cite{res21}
~\cite{res22}

\section{Getting started}

CODE

\begin{verbatim}
for(i : )
    do
\end{verbatim}

\LaTeX{} version:

\begin{munquote}[~\cite{res1}]%
qoteee%
\end{munquote}



By default, all text is double spaced, however, quotes and footnotes
abels to refer to a page.  For example, Chapter~\ref{chap:conclusions}
starts on page~\pageref{chap:conclusions}.

\section{Some Suggestions}

Here are a few recommendations:

\begin{itemize}
	\item Before using this template, make sure you check with
		your supervisor.
	\item Concentrate on content, let \LaTeX{} handle the typesetting.
	\item Don't worry about warnings related to:
	\begin{itemize}
		\item overfull \texttt{hboxes}/\texttt{boxes}
		\item underfull \texttt{hboxes}/\texttt{vboxes}
	\end{itemize}
	These can be corrected with modest rewording of your text prior
	to submission of your final copy.
\end{itemize}

\section{The \texttt{Makefile}}

You can use \texttt{make} to ``build'' your thesis on the Linux command
line\munfootnote{Linux is available on all machines running LabNet in
\textsl{The Commons} and in other computer labs on campus.} This will
automatically run the \texttt{bibtex} program

\subsubsection{Subsubsection}
\textsf{$<$Empty subsection$>$}


-to collect in one research a set of methods of functional code refactoring using design patterns
-to find out whether mathematical theory can be applied to the most "mathematical" programming paradigm with aim to create common design solutions/optimise functional code
-to make a comparison of well-known design patterns from object-oriented paradigm to functional approach
-to create utility for making XSLT transformations which is missed in current Erlang/OTP distribution

