\addcontentsline{toc}{chapter}{Abstract}
\begin{center}
\textbf{\large Abstract}
\end{center}

The purpose of this study was to collect in one research a set of methods of functional code refactoring using design patterns.
Another aim was to find out whether mathematical theory can be applied to the most "mathematical" programming paradigm with aim to create common design solutions/optimise functional code. Finally, comparison of well-known design patterns from object-oriented paradigm to functional approach was done in the thesis.

The results of the study were organised as a set of patterns and refactoring methods and results of their application in project - XSLT transformer.

It was discovered that functional paradigm does not have most of problems that OOP patterns try to solve. As a result - applications written in functional languages could be much less complex and verbose than same ones in OOP.

The principal conclusion was that it is much easier to write/maintain software written in pure functional languages. It is the result of the fact that we don't need massive amount of code to realise pattern-like behaviour.

\vspace{1cm}
